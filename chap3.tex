\chapter{先行研究:歩行情報蓄積基盤}
\thispagestyle{myheadings}

\section{歩行情報蓄積基盤の概要}
我々は,PDR(歩行者自律航法)やマップマッチング,パーティクルフィルタ等を組み合わせて,移動データを取得し蓄積する技術(DB)を構築した[3].図1.2にシステム概要図を示す.本システムは歩行者端末,歩行情報蓄積サーバ,軌跡・歩行情報・フロアマップDB の三要素から構成される.まず,歩行者端末がステップ検出を行ったタイミングでセンサデータのアップロードを行う.その後,歩行情報蓄積サーバは受信したセンサデータを元に位置推定を行い,センサデータ及びその推定結果をDBに蓄積する .また,軌跡・歩行情報・フロアマップDB には歩行可能・不可能領域が示されたフロアマップと電波フィンガープリントが事前に入力されている. これらの情報に基づいてパーティクルフィルタによる位置推定を行う.

\begin{figure}[htb]
    \centering
    \includegraphics[width=\linewidth]{img/pedestrian_data_platform.jpg
    }
    \caption{先行研究:歩行情報蓄積基盤}
    \label{fig:}
\end{figure}

\section{歩行者端末}

\section{歩行軌跡生成・歩行情報蓄積サーバ}


\section{歩行軌跡・歩行情報・フロア情報DB とオブジェクトストレージ}


%TODO:コピペ






\chapter{はじめに}
\thispagestyle{myheadings}
本章では,研究背景や我々の先行研究について述べ,本研究の目的と概要を説明する.

\section{背景}
\label{sec:abstract}

\subsection{実店舗における行動分析の重要性}
近年,大型ショッピングモールなどの商業施設において,売上向上や顧客満足度の改善を目的としたデータ分析の需要が高まっている.
特に,店舗内の動線を定量的に把握することは,以下の点において重要である.1点目は,レイアウトの最適化である. 顧客が通過しやすい通路や,立ち止まりやすいエリアを特定し,商品配置の最適化が可能である.2点目は,施策の効果測定である. キャンペーンやデジタルサイネージの設置などが,実際の顧客行動にどのような変容をもたらしたか客観的な評価が可能である.3点目は.業務効率化である. 混雑状況に応じたスタッフの配置や空調制御など,運営コストの削減に寄与する.

一方で,ECサイトでは閲覧履歴やクリック率,滞在時間などの詳細なログデータを取得し,それに基づいたWebデザインの改善やレコメンデーションシステムによる購買率の向上が一般的に行われている.実店舗においては,POSデータによる購買結果の分析は行われているものの,購買に至るまでの「回避行動」や「商品前での滞在」といったプロセスの可視化は,ECサイトに比べて遅れているのが現状である.経験や勘に頼っていた店舗運営から,データに基づく意思決定への転換が求められている.

\subsection{屋内測位技術と課題}
人の動線の分析を行うためには高精度な人流データの取得が必要である.屋外環境においてはGPS(Global Positioning System)が広く利用されている.GPSの欠点は,鉄筋コンクリートなどの遮蔽物に弱く,屋内や地下街では正確な位置特定が困難である.そのため,屋内環境における位置測位の研究が盛んに行われている.

代表的な屋内測位手法として,Wi-FiやBLE(Bluetooth Low Energy)ビーコンを用いた手法が挙げられる.これらは受信強度(RSSI:Received Signal Strength Indicator)を用いて位置を推定するものである.事前に計測した電波マップと照合するフィンガープリント方式は比較的高い精度を実現できる.一方で,スマートフォンに内蔵された加速度センサやジャイロセンサを用いて移動量を推定するPDR(Pedestrian Dead Reckoning:歩行者自律航法)は,専用のインフラ整備を必要とせず,低コストで導入可能である利点がる.ただし,PDRはセンサの誤差が累積するため,長時間・長距離の歩行に伴い,位置推定結果の精度が低い問題がある.そのため,PDRと\cite{mizutani}.

%TODO:明日はここから!!

屋内測位には様々な手法が存在し,代表的な測位手法としてフィンガープリント手法がある.フィンガープリントWi-Fi,BLEなどの無線通信技術を.事前に電波強度の情報を元に,現在位置の観測データと照合して位置を推定する.フィンガープリント法の問題点として,電波強度の収集に時間的コストがかかる点がある.


屋内施設のレイアウト変更などの施策が適切か判断するためには,施策前後で滞在状況の変化を把握する必要がある.実店舗では売上向上のために施設内のレイアウト変更の施策が行われている.施策が適切か判断するためには,施策前後で顧客の滞在状況が変化したか判断する必要がある.



\section{目的とアプローチ}
\label{sec:thesis}
本研究は屋内施設の滞在状況の変化を可視化する滞在比較システムを提案する.アプローチとして施策前後の複数の移動データから滞在箇所を抽出.施設前後の滞在回数の変化量を差分ヒートマップで可視化する.図\ref{fig:research_outline}に本研究の概要を示す.

\begin{figure}[htb]
    \centering
    \includegraphics[width=\linewidth]{img/research_outline.jpg}
    \caption{本研究の概要図}
    \label{fig:research_outline}
\end{figure}



\section{論文構成}
\label{sec:presentation}
本稿の構成は以下の通りである.2章では,行動分析に関する既存研究を紹介し,その利点や問題点を述べる.3章では,屋内施設の滞在状況の変化を可視化する滞在比較システムについて述べる.4章では,屋内施設の滞在状況の変化を可視化する滞在比較システムの評価,考察を行う.最後に5章で,本研究のまとめを行う.

% Local Variables: 
% mode: japanese-LaTeX
% TeX-master: "root"
% End: 

\chapter*{謝辞}
\addcontentsline{toc}{chapter}{\protect\numberline {} 謝辞}

本研究を進めるにあたり,多くの御指導,御鞭撻を賜わりました,愛知工業大学の教授の皆様に深く感謝致します.

また,御討論、御助言していただきました,愛知工業大学,梶克彦教授に深く感謝致します.

最後に,日頃から熱心に討論,助言してくださいました.研究室のみなさんに深く感謝致します.

\paragraph{あとがき〜研究室配属からの2年間を語る}
本論文を締めくくるにあたり,研究室での活動を振り返り,感謝の意を表します.

研究室配属後の2年間は,私にとって非常に充実した時間となりました.

特に印象に残っているのは,3年次の夏に研究室のメンバー4人で挑戦したハッカソンです.自分の担当は,バックエンド開発とプレゼンテーションを担当しました.当初は,自分の技術不足に不安を感じていました.しかし,先輩やチームメンバの手厚いサポートを受けながら開発を進めました.結果は,20チーム中1チームに選ばれる「優秀賞」をいただきました.この経験は,一人だけの力ではなし得ない.「チームプレーの可能性」を体感しました.また,システム開発の面白さに気づきました.この経験は,システム開発への不安を払拭し,「エンジニアの道に進もう」決意を固める大きな転換点となりました.

4年次の卒業研究では,自分1人の力で研究を進める難しさに直面しました.その過程で,多くのソフトスキルが身につきました.ソフトスキルとは,論理的思考力やコミュニケーション能力です.具体的には,日々の進捗報告で目標設定やタスク管理,そして自分の考えを論理的に伝え議論を深めるコミュニケーション能力です.これらのスキルは,日々の研究の積み重ねによって得られた一生の財産です.技術的なハードスキルだけでなく,これらのソフトスキルは,今後の社会人生活において大きな糧になると感じています.

最後になりますが,熱心にご指導をしてくれた梶克彦教授,ならびに切磋琢磨し支え合ってきた研究室の皆さんに心より感謝いたします.春からはそれぞれ異なる道を歩むことになりますが,研究室での経験を胸に日々精進して参ります.研究室に残る皆さん,そして卒業される皆さんの今後のご健勝とご多幸を心よりお祈り致します.お互いに体調に気をつけながら頑張りましよう.





% Local Variables: 
% mode: latex
% TeX-master: "root"
% End: 

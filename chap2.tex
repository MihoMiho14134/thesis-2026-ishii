\thispagestyle{myheadings}
\chapter{関連研究}
\label{sec:format}

\section{屋内側位技術に関する関連研究}
\label{sec:style}

屋内測位には様々な手法が存在する.ここではPDRと電波フィンガープリント,マップマッチングを用いて屋内測位を行う研究を紹介し,それらをパーティクルフィルタにより統合するハイブリッド型屋内測位手法を行っている研究について述べる.

\subsection{PDR}

???
PDRとは,歩行者に身につけられたセンサから得られる加速度を角速度などの歩行データをもとに,初期地点からの歩行距離および進行方向を予測し,現在位置を推定する方法である.[若林の??].PDRの利点は,一般的にWi-FiアクセスぽいんとやBluetoothビーコンなどの外部インフラを利用しないため,導入コストが低い点である.欠点は,累積誤差.図?で,累積誤差によりPDRの歩行軌跡を示す.

\subsection{電波フィンガープリント}
電波フィンガープリントとは,


\subsection{マップマッチング}

\subsection{ハイブリッド型屋内側位手法}


\section{歩行軌跡を行動分析に用いた研究}
\label{sec:walking_analysis}
研究[??]では,既に蓄積されたセンサデータから歩行速度と軌跡を導出し,活用する行動分析の手法を提案している.具体的には,一定期間に取得された歩行速度の加速,減速具合から休憩中,買い物中など歩行者の行動を整理・分類をする.そして,分類した歩行者の行動と歩行軌跡と地図上の位置を活用して,動線を元にした空間デザインの最適化のガイドラインを提案している.これらの研究より,歩行軌跡は屋内測位精度の維持・向上や空間の利用状況把握などに有用な情報であり,蓄積する価値があると考える.

人は日常生活の約87\% の屋内で過ごすとされており,屋内施設における行動分析は有用である.滞在行動をに関する都市スケールの研究では,山口らがクレジットカードデータとGPS滞在履歴を用いて、COVID-19前後における名古屋市内の滞在・消費行動の時空間的な変化を分析している[2].これに対し本研究では,分析対象の規模を屋内施設内に限定し,施策による滞在状況の変化を分析する.
 

指標を滞在状況にした理由は,レイアウト変更などの施策の効果の測定をする際に,滞在した時間と顧客の関心度が相関し,関心度を測定する情報として重要なためである.宮崎らは滞在時間と顧客の関心度が相関するを示している[1].
したがって,レイアウト変更などの施策の効果を測るには通行量を見るだけではなく,滞在時間などの滞在状況を捉える必要がある.

\section{比較可視化する関連研究}
\label{sec:comparative_visualization}
データの比較や変化の抽出を行う手法は,他分野でも研究されている.例えば,教育支援やスキル評価の分野では,模範データとユーザデータの差分を可視化する研究がある.松井らの研究は,ピアノの演奏の評価において,模範演奏と学習者の演奏を画像化し,異常検知技術を用いて,その差分をヒートマップとして可視化する手法を提案している[??].差分を色で強調表示することで,ユーザの演奏の違いを直感的に把握できる.


\clearpage

% Local Variables: 
% mode: japanese-LaTeX
% TeX-master: "root"
% End: 
